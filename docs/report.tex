\documentclass{article}[letterpaper, 11pt]
\usepackage{geometry}
\geometry{left=0.75in,top=0.75in,bottom = 0.75in,right=0.75in}
\usepackage[utf8]{inputenc}
\usepackage{amsmath}
\usepackage{bbm}
\usepackage{amssymb}
\usepackage{enumitem}

\usepackage{lastpage}
\usepackage{fancyhdr}

\usepackage{svg}
\usepackage{graphicx}
\graphicspath{ {images/} }

\usepackage[parfill]{parskip}

\usepackage{color}   %May be necessary if you want to color links
\usepackage{hyperref}
\hypersetup{
    colorlinks=true, %set true if you want colored links
    linktoc=all,     %set to all if you want both sections and subsections linked
    linkcolor=black,  %choose some color if you want links to stand out
}

%Style :
\pagestyle{fancy}
\renewcommand{\thepage}{}
%commancer le compte des pages après
\setlength{\headsep}{14.5pt}

%Style nouveau chapitre
\fancyhf{}
\fancyhead[L]{\small INF8770 --- Technologies multimédias}
\fancyhead[C]{\small TP2 --- Compression transformée KL}
\fancyhead[R]{\small Page \thepage\hspace{1pt} sur~\pageref{LastPage}}
\fancyfoot[L]{\scriptsize Polytechnique Montréal}
\fancyfoot[C]{\scriptsize Matthieu Basset, Hiver 2024}
\fancyfoot[R]{\scriptsize Département de GIGL}
\renewcommand{\headrulewidth}{0.5pt}
\renewcommand{\footrulewidth}{0.2pt}

\title{
    \includegraphics[scale=0.8]{poly-logo.pdf}\\\vspace*{50pt}
    \huge\textbf{INF8770}\\
    \textbf{Technologies Multimédias}\\
    Travail pratique \#2\\
    Compression d'images avec la transformée \\Karhunen-Loève\\
}

\author{\Large Matthieu Basset --- 2225981}
\date{\huge Janvier 2024}
\begin{document}

\thispagestyle{empty}
\maketitle

\newpage

\renewcommand{\thepage}{\arabic{page}}
\pagestyle{fancy}
\renewcommand{\contentsname}{Table des matières}
\setcounter{page}{1}

\tableofcontents


\newpage

\section{Question 1}


\newpage
\section{Question 2}


\begin{itemize}[label={--}]
	\item Liste label tirets
\end{itemize}


\newpage
\section{Question 3}

\newpage
\section{Question 4}
\[
\begin{array}{|l|c|} \hline % chktex 44
Tableau & oui\\ \hline % chktex 44
\end{array}\]

% Deux images:

% \begin{figure}[h]
% 	\begin{minipage}[c]{.49\linewidth}
% 		 \centering
% 		 \includegraphics[scale = 0.25]{poly-logo.pdf}
% 	\end{minipage}
% 	\begin{minipage}[c]{.49\linewidth}
% 		 \centering
% 		 \includegraphics[scale = 0.25]{poly-logo.pdf}
% 	\end{minipage}
% 		 \caption{ \centering Deux fois le logo de Poly, ouais.}
% \end{figure}

% Une magnifique liste à tirets:
% \begin{itemize}[label={--}]
% 	\item Un
% 	\item Deux
% 	\item Trois
% \end{itemize}
% Une autre liste, cette fois avec des chiffres
% \begin{enumerate}[label={E\arabic*.}]
% 	\item item
% 	\item item
% 	\item item
% \end{enumerate}

% Une liste qui suit les numéros (wow)

% \begin{enumerate}[label={E\arabic*.}, resume]
% 	\item item
% 	\item item
% \end{enumerate}

% \newpage
% \section*{ANNEXES}

% \setcounter{section}{0}
% \renewcommand{\thesection}{\Alph{section}}
% \section{Première section d'annexe}
% Des trucs, des refs??

% \newpage
% \section{Formules utilisées}\label{sec:truc}
% \subsection{Autre annexe}
% Texte
\end{document}
